\section*{Esercizio 4}

La classe {\tt SolutionsEnumerator} (\autoref{lst:enum}) estende la precedente
classe ed implementa l'interfaccia {\tt Enumeration}. Il metodo {\tt
searchNext()} utilizza lo stesso principio della ricerca in backtracking ma in
modo iterativo: una volta effettuata la propagazione dei vincoli con il metodo
{\tt infer()}, ereditato da {\tt Solver}, i nuovi dominî (e rispettivi
iteratori) vengono memorizzati in una pila e il metodo {\tt searchNext()} lavora
unicamente sugli ultimi domini inseriti. Una volta trovata una soluzione, se il
metodo viene successivamente richiamato, questo riprende la ricerca dall'ultimo
elemento elaborato senza dover eseguire nuovamente la propagazione dei vincoli.

\lstinputlisting[caption=SolutionsEnumerator.java, label=lst:enum]{../../../src/dsl/SolutionsEnumerator.java}
