\section*{Esercizio 5}

Vengono estese le classi {\tt Relation} e derivate con il metodo ({\tt
protected}) {\tt explain()} (\autoref{lst:rel_c}, \ref{lst:eq_c} e
\ref{lst:diff_c}). Questi metodi, data una coppia di valori appartenente alla
relazione, restituiscono una stringa contenente le restrizioni sul codominio.
Nel caso di {\tt EqConstraint} l'unico vincolo rilevante non può che essere
quello passato per argomento, mentre per {\tt DiffConstraint} viene effettuata
una concatenazione dei vari vincoli memorizzati nell'afterset. Se la relazione è
invece {\tt Relation}, significa che non è stato espresso alcun vincolo e quindi
non viene fornita alcuna motivazione. Viene dunque estesa anche la classe {\tt
Solver} con il metodo {\tt explain()} (\autoref{lst:solver_c}), questa volta
pubblico, che restituisce una motivazione ({\tt String}) all'ultima soluzione
fornita (sia da {\tt backtrackingSearch()} che da {\tt searchNext()}, nel caso
di {\tt SolutionsEnumerator}). Questo metodo non fa altro che concatenare
elegantemente le motivazioni fornite dai metodi {\tt explain()} dei vari oggetti
{\tt Relation}. Un esempio di stringa restituita da questo metodo verrà fornita
nel prossimo esercizio\footnote{Essendo le motivazioni fornite solamente a
soluzioni generate dalle classi {\tt Solver} e {\tt SolutionsEnumerator}, non
occorre effettuare controlli sulle coppie di valori passate come argomento ai
metodi {\tt explain()} delle classi {\tt Relation} e derivate, poiché saranno
sicuramente presenti nei loro afterset. Questo è inoltre possibile poiché la
visibilità di questi metodi è {\tt protected}, quindi ristretta al package e
alle sottoclassi.}.
%
\begin{lstlisting}[caption=Relation.java (cont.), label=lst:rel_c, firstnumber=30]
    protected String explain(Value lValue, Value rValue) { return ""; }
\end{lstlisting}
%
\begin{lstlisting}[caption=EqConstraint.java (cont.), label=lst:eq_c, firstnumber=15]
    @Override
    protected String explain(Value lValue, Value rValue) {
        return rValue.getDomainName() + " = " + rValue;
    }
\end{lstlisting}
%
\begin{lstlisting}[caption=DiffConstraint.java (cont.), label=lst:diff_c, firstnumber=16]
    @Override
    protected String explain(Value lValue, Value rValue) {
        StringBuilder builder = new StringBuilder();
        getAfterSet(lValue).forEach(v -> builder.append(" \u2227 ") // AND
                .append(v.getDomainName()).append(" \u2260 ").append(v)); // NEQ
        builder.delete(0, 3);

        return builder.toString();
    }
\end{lstlisting}
%
\begin{lstlisting}[caption=Solver.java (cont.), label=lst:solver_c, firstnumber=59]
    public String explain() {
        if (solution.isEmpty())
            return "Solution not found.";

        StringBuilder builder = new StringBuilder();

        for (int i = 0; i < sets.size(); i++) {
            builder.append(sets.get(i)).append(" = ")
                    .append(solution.get(i)).append("  \u21D2  "); // IMPLIES

            for (int j = 0; j < sets.size(); j++) {
                if (i == j)
                    continue;

                String pairs = sets.get(i).getRelationWith(sets.get(j))
                        .explain(solution.get(i), solution.get(j));

                if (!pairs.equals(""))
                    builder.append(pairs).append(" \u2227 "); // AND
            }

            builder.delete(builder.length() - 3, builder.length()).append("\n");
        }

        return builder.toString();
    }
\end{lstlisting}
