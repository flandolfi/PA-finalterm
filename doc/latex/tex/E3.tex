\section*{Esercizio 3}

La classe {\tt Solver} (\autoref{lst:solver}) implementa una ricerca in
backtracking\footnote{Riferimenti: S. Russel, P. Norvig. \emph{Artificial
Intelligence: A Modern Approach (Third Edition)}. 2010.} (ricorsiva). Ad ogni
assegnazione di valore viene effettuata una propagazione dei vincoli tramite il
metodo {\tt infer()}.

% \lstinputlisting[caption=Solver.java, label=lst:solver]{../../../src/solver/Solver.java}
\bgroup
\createlinenumber{58}{\smallvdots}
\createlinenumber{59}{103}
\begin{lstlisting}[caption=Solver.java, label=lst:solver]
package dsl;

import compiler.*;
import java.io.*;
import java.util.*;
import java.util.stream.Collectors;

public class Solver {
    protected LinkedList<Value> solution = new LinkedList<>();
    protected LinkedList<Set<Value>> universe;
    protected ArrayList<Domain> sets;

    public Solver(Collection<Domain> sets) {
        this.sets = new ArrayList<>(sets);
        universe = sets.stream().map(Domain::getValues)
                .collect(Collectors.toCollection(LinkedList::new));
    }

    public List<Value> backtrackingSearch() {
        solution = backtrack(new LinkedList<>(), universe);

        return solution;
    }

    protected LinkedList<Value> backtrack(LinkedList<Value> solution,
                                        LinkedList<Set<Value>> prevInferences) {
        return solution.size() == sets.size()? solution : prevInferences.getFirst()
                .stream().map(v -> infer(v, solution, prevInferences))
                .filter(Objects::nonNull).map(i -> backtrack(solution, i))
                .filter(s -> s != null || solution.removeLast() == null)
                .findFirst().orElse(null);
    }

    protected LinkedList<Set<Value>> infer(Value value,
                LinkedList<Value> solution, LinkedList<Set<Value>> inferences) {
        int step = solution.size();

        for (int i = 0; i < step; i++)
            if (!sets.get(step).getRelationWith(sets.get(i))
                    .getAdjacencySet(value).contains(solution.get(i)))
                return null;

        LinkedList<Set<Value>> result = new LinkedList<>();

        for (int i = step + 1; i < sets.size(); i++) {
            result.add(new HashSet<>(inferences.get(i - step)));
            result.getLast().retainAll(sets.get(step)
                    .getRelationWith(sets.get(i)).getAdjacencySet(value));

            if (result.getLast().isEmpty())
                return null;
        }

        solution.add(value);

        return result;
    }
    // See ex. 5 & 6
}
\end{lstlisting}
\egroup
