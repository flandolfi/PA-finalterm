\section*{Esercizio 6}

Nel file {\bf Sudoku.csp} (\autoref{lst:sudoku}) viene espresso con il DSL del
progetto un sudoku di ordine 2 con configurazione iniziale come in
\autoref{fig:before}. Estendendo ulteriormente la classe {\tt Solver} con il
metodo {\tt main()} ed eseguendolo con il percorso del file {\bf Sudoku.csp}
come argomento, otterremo il seguente output (rappresentabile come in
\autoref{fig:after}):
%
\begin{lstlisting}[mathescape=true, numbers=none, frame=none]
$\text{\tt AA} = \text{\tt AA1}  \:\Rightarrow\:  \text{\tt BB} \neq
\text{\tt BB1} \land \text{\tt AB} \neq \text{\tt AB1} \land \text{\tt AC} \neq
\text{\tt AC1} \land \text{\tt AD} \neq \text{\tt AD1} \land \text{\tt DA} \neq
\text{\tt DA1} \land \text{\tt CA} \neq \text{\tt CA1} \land \text{\tt BA} \neq
\text{\tt BA1}$
$\text{\tt CC} = \text{\tt CC1}  \:\Rightarrow\:  \text{\tt DD} \neq
\text{\tt DD1} \land \text{\tt CD} \neq \text{\tt CD1} \land \text{\tt BC} \neq
\text{\tt BC1} \land \text{\tt AC} \neq \text{\tt AC1} \land \text{\tt CA} \neq
\text{\tt CA1} \land \text{\tt CB} \neq \text{\tt CB1} \land \text{\tt DC} \neq
\text{\tt DC1}$
$\text{\tt BB} = \text{\tt BB2}  \:\Rightarrow\:  \text{\tt AB} \neq
\text{\tt AB2} \land \text{\tt BC} \neq \text{\tt BC2} \land \text{\tt CB} \neq
\text{\tt CB2} \land \text{\tt BA} \neq \text{\tt BA2}$
$\text{\tt DD} = \text{\tt DD3}  \:\Rightarrow\:  \text{\tt CD} \neq
\text{\tt CD3} \land \text{\tt AD} \neq \text{\tt AD3} \land \text{\tt DA} \neq
\text{\tt DA3} \land \text{\tt DC} \neq \text{\tt DC3}$
$\text{\tt AB} = \text{\tt AB3}  \:\Rightarrow\:  \text{\tt BB} \neq
\text{\tt BB3} \land \text{\tt AC} \neq \text{\tt AC3} \land \text{\tt AD} \neq
\text{\tt AD3} \land \text{\tt CB} \neq \text{\tt CB3} \land \text{\tt BA} \neq
\text{\tt BA3}$
$\text{\tt CD} = \text{\tt CD2}  \:\Rightarrow\:  \text{\tt DD} \neq
\text{\tt DD2} \land \text{\tt AD} \neq \text{\tt AD2} \land \text{\tt CA} \neq
\text{\tt CA2} \land \text{\tt CB} \neq \text{\tt CB2} \land \text{\tt DC} \neq
\text{\tt DC2}$
$\text{\tt BC} = \text{\tt BC3}  \:\Rightarrow\:  \text{\tt BB} \neq
\text{\tt BB3} \land \text{\tt AC} \neq \text{\tt AC3} \land \text{\tt AD} \neq
\text{\tt AD3} \land \text{\tt BA} \neq \text{\tt BA3} \land \text{\tt DC} \neq
\text{\tt DC3}$
$\text{\tt BD} = \text{\tt BD1}  \:\Rightarrow\:  \text{\tt BB} \neq
\text{\tt BB1} \land \text{\tt DD} \neq \text{\tt DD1} \land \text{\tt CD} \neq
\text{\tt CD1} \land \text{\tt BC} \neq \text{\tt BC1} \land \text{\tt AC} \neq
\text{\tt AC1} \land \text{\tt AD} \neq \text{\tt AD1} \land \text{\tt BA} \neq
\text{\tt BA1}$
$\text{\tt AC} = \text{\tt AC2}  \:\Rightarrow\:  \text{\tt AB} \neq
\text{\tt AB2} \land \text{\tt BC} \neq \text{\tt BC2} \land \text{\tt AD} \neq
\text{\tt AD2} \land \text{\tt DC} \neq \text{\tt DC2}$
$\text{\tt AD} = \text{\tt AD4}  \:\Rightarrow\:  \text{\tt DD} \neq
\text{\tt DD4} \land \text{\tt AB} \neq \text{\tt AB4} \land \text{\tt CD} \neq
\text{\tt CD4} \land \text{\tt BC} \neq \text{\tt BC4} \land \text{\tt AC} \neq
\text{\tt AC4}$
$\text{\tt DA} = \text{\tt DA2}  \:\Rightarrow\:  \text{\tt DD} \neq
\text{\tt DD2} \land \text{\tt CA} \neq \text{\tt CA2} \land \text{\tt CB} \neq
\text{\tt CB2} \land \text{\tt BA} \neq \text{\tt BA2} \land \text{\tt DC} \neq
\text{\tt DC2}$
$\text{\tt DB} = \text{\tt DB1}  \:\Rightarrow\:  \text{\tt BB} \neq
\text{\tt BB1} \land \text{\tt DD} \neq \text{\tt DD1} \land \text{\tt AB} \neq
\text{\tt AB1} \land \text{\tt DA} \neq \text{\tt DA1} \land \text{\tt CA} \neq
\text{\tt CA1} \land \text{\tt CB} \neq \text{\tt CB1} \land \text{\tt DC} \neq
\text{\tt DC1}$
$\text{\tt CA} = \text{\tt CA3}  \:\Rightarrow\:  \text{\tt CD} \neq
\text{\tt CD3} \land \text{\tt DA} \neq \text{\tt DA3} \land \text{\tt CB} \neq
\text{\tt CB3} \land \text{\tt BA} \neq \text{\tt BA3}$
$\text{\tt CB} = \text{\tt CB4}  \:\Rightarrow\:  \text{\tt BB} \neq
\text{\tt BB4} \land \text{\tt AB} \neq \text{\tt AB4} \land \text{\tt CD} \neq
\text{\tt CD4} \land \text{\tt DA} \neq \text{\tt DA4} \land \text{\tt CA} \neq
\text{\tt CA4}$
$\text{\tt BA} = \text{\tt BA4}  \:\Rightarrow\:  \text{\tt BB} \neq
\text{\tt BB4} \land \text{\tt AB} \neq \text{\tt AB4} \land \text{\tt BC} \neq
\text{\tt BC4} \land \text{\tt DA} \neq \text{\tt DA4} \land \text{\tt CA} \neq
\text{\tt CA4}$
$\text{\tt DC} = \text{\tt DC4}  \:\Rightarrow\:  \text{\tt DD} \neq
\text{\tt DD4} \land \text{\tt CD} \neq \text{\tt CD4} \land \text{\tt BC} \neq
\text{\tt BC4} \land \text{\tt AC} \neq \text{\tt AC4} \land \text{\tt DA} \neq
\text{\tt DA4}$
\end{lstlisting}

\begin{figure}[t]
    \centering
    \setlength\arrayrulewidth{0.8pt}
    \setlength\extrarowheight{3pt}
    \setlength{\aboverulesep}{0pt}
    \setlength{\belowrulesep}{0pt}
    \begin{subfigure}{0.3\textwidth}
        \centering
        \begin{tabular}{m{0.1cm}|m{0.1cm}!{\vrule width 0.1pt}m{0.1cm}|m{0.1cm}%
            !{\vrule width 0.1pt}m{0.1cm}|m{0.1cm}}
            \multicolumn{1}{c}{} &
            \multicolumn{1}{c}{\scshape{a}} &
            \multicolumn{1}{c}{\scshape{b}} &
            \multicolumn{1}{c}{\scshape{c}} &
            \multicolumn{1}{c}{\scshape{d}} & \\
            \cline{2-5}
            \scshape{a} & {\bf 1} & & & & \\
            \cmidrule[0.1pt]{2-5}
            \scshape{b} & & & & {\bf 1} & \\
            \cline{2-5}
            \scshape{c} & & & {\bf 1} & & \\
            \cmidrule[0.1pt]{2-5}
            \scshape{d} & & {\bf 1} & & & \\
            \cline{2-5}
        \end{tabular}
        \caption{} \label{fig:before}
    \end{subfigure}
    \begin{subfigure}{0.3\textwidth}
        \centering
        \begin{tabular}{m{0.1cm}|m{0.1cm}!{\vrule width 0.1pt}m{0.1cm}|m{0.1cm}%
            !{\vrule width 0.1pt}m{0.1cm}|m{0.1cm}}
            \multicolumn{1}{c}{} &
            \multicolumn{1}{c}{\scshape{a}} &
            \multicolumn{1}{c}{\scshape{b}} &
            \multicolumn{1}{c}{\scshape{c}} &
            \multicolumn{1}{c}{\scshape{d}} &\\
            \cline{2-5}
            \scshape{a} & {\bf 1} & 3 & 2 & 4 & \\
            \cmidrule[0.1pt]{2-5}
            \scshape{b} & 4 & 2 & 3 & {\bf 1} & \\
            \cline{2-5}
            \scshape{c} & 3 & 4 & {\bf 1} & 2 &\\
            \cmidrule[0.1pt]{2-5}
            \scshape{d} & 2 & {\bf 1} & 4 & 3 & \\
            \cline{2-5}
        \end{tabular}
        \caption{} \label{fig:after}
    \end{subfigure}
    \caption{Modello del Sudoku utilizzato: {\bf (a)} stato iniziale; {\bf (b)}
    risolto con il metodo {\tt backtrackingSearch()}.}
\end{figure}

\lstinputlisting[caption=Sudoku.csp, label=lst:sudoku,%
tabsize=4]{../../../test/samples/Sudoku.csp}
%
\begin{lstlisting}[caption=Solver.java (cont.), label=solver_c2, firstnumber=86]
    public static void main(String[] args) {
        if (args.length == 0) {
            System.err.println("ERROR: Missing argument: path to (.csp) file.");
            System.exit(1);
        }

        try {
            Solver solver = new Solver((new Parser()).parse(
                    new BufferedReader(new FileReader(args[0]))));
            solver.backtrackingSearch();
            System.out.println(solver.explain());
        } catch (CompilerException e) {
            System.err.println("ERROR: File " + args[0] + " @ " + e.getMessage());
        } catch (FileNotFoundException e) {
            System.err.println("ERROR: File " + args[0] + " not found.");
        }
    }
\end{lstlisting}
