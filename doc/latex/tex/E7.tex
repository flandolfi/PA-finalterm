\section*{Esercizio 7}

La \emph{reflection} è la capacità di un programma di esaminare i suoi stessi
metadati e, nel caso dei linguaggi di programmazione a oggeti, di: %
%
\begin{enumerate*}
    \item ispezionare metodi e attributi di classi ed interfacce;
    \item accedere e modificare attributi di oggetti e classi;
    \item invocare metodi di oggetti e classi.
\end{enumerate*}
%
Questo è possibile fintanto che viene mantenuta una tabella dei simboli
(\emph{symbol table}), ovvero una struttura dati che mappa ogni identificatore
alle sue relative informazioni conosciute. Alcuni dei principali linguaggi
dotati di reflection sono: Java, C\# (e gli altri linguaggi .NET), Perl, PHP,
Tcl, Python, Ruby e JavaScript. La reflection viene usata principalmente per
creare programmi che manipolano altri programmi, la quale struttura può non
essere conosciuta a priori. Può essere utile, ad esempio, per effettuare test su
codice. Spesso, invece, risulta più conveniente l'utilizzo di polimorfismo. Ad
esempio, nel metodo {\tt explain()} della classe {\tt Solver}
(\autoref{lst:solver_c}) avrei potuto generare la stringa in base al tipo di
vincolo trovato (ad es. \texttt{if(R \textbf{instaceof} EqConstraint) \{} ...
{\tt \}}). Nonostate questo tipo di programmazione sia comune nei linguaggi di
scripting, nei linguaggi di programmazione ad oggetti è più appropriato (e più
efficiente) utilizzare il polimorfismo ed eseguire un metodo comune alle
sottoclassi
%
\footnote{Riferimenti:%
\begin{itemize}[noitemsep,topsep=0pt]
    \item[--] M. L. Scott. \emph{Programming Language Pragmatics (Third
    Edition)}. 2009.
    \item[--] C. Szyperski, D. Gruntz, S. Murer. \emph{Component Software --
    Beyond Object Oriented Programming (Second Edition)}. 2002.
    \item[--] Oracle.
    \href{http://docs.oracle.com/javase/tutorial/reflect/index.html}{
    \emph{Java\texttrademark{} Documentation. Trail: The Reflection API}}. 2015.
\end{itemize}}.
