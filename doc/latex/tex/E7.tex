\section*{Esercizio 7}

La \emph{reflection} è la capacità di un programma di manipolare la sua
struttura a tempo di esecuzione. Nel caso dei linguaggi di programmazione a oggetti, questo permette di: %
%
\begin{enumerate*}
    \item ispezionare metodi e attributi di classi ed interfacce;
    \item accedere e modificare attributi di oggetti e classi;
    \item invocare metodi di oggetti e classi.
\end{enumerate*}
%
La struttura deve poter essere esprimibile nello stesso linguaggio del programma
sotto forma di dati (\emph{metadati}) o oggetti (\emph{metaoggetti}) che ne
descrivano gli aspetti.
%
Alcuni dei principali linguaggi dotati di reflection sono: Java, C\# (e gli
altri linguaggi .NET), Perl, PHP, Tcl, Python, Ruby e JavaScript. La reflection
viene usata principalmente per creare programmi che manipolano altri programmi,
la quale struttura può non essere conosciuta a priori. Può essere utile, ad
esempio, per effettuare debug o test di codice. Spesso, in OOP, la reflection
può essere sostituita dal polimorfismo. Ad esempio, nel metodo {\tt explain()}
della classe {\tt Solver} (\autoref{lst:solver_c}) avremmo potuto generare la
stringa in base al tipo di vincolo (ad es. \texttt{if(R \textbf{instaceof}
EqConstraint) \{} ... {\tt \}}). Nonostate questo modello di programmazione sia
comune in alcuni linguaggi di scripting, in OOP è più appropriato (e più
efficiente) eseguire un metodo di un'interfaccia o di una superclasse comune e
variarne il comportamento in base alla sottoclasse, come ad esempio {\tt
explain()} della classe {\tt Relation} e derivate (si veda
l'\hyperref[sec:ex5]{Esercizio 5})
%
\footnote{Riferimenti:%
\begin{itemize}[noitemsep,topsep=0pt]
    \item[--] K. Czarnecki, U. W. Eisenecker. \emph{Generative Programming:
    Methods, Tools, and Applications}. 2000.
    \item[--] Oracle.
    \href{http://docs.oracle.com/javase/tutorial/reflect/index.html}{
    \emph{Java\texttrademark{} Documentation. Trail: The Reflection API}}. 2015.
    \item[--] M. L. Scott. \emph{Programming Language Pragmatics (Third
    Edition)}. 2009.
    \item[--] C. Szyperski, D. Gruntz, S. Murer. \emph{Component Software --
    Beyond Object Oriented Programming (Second Edition)}. 2002.
\end{itemize}}.
