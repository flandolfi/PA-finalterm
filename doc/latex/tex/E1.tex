\section*{Esercizio 1}

Nel package {\tt dsl} (\autoref{lst:val}, \ref{lst:set}, \ref{lst:rel},
\ref{lst:eq} e \ref{lst:diff}) sono contenute le classi utilizzate per
rappresentare il DSL. La classe \texttt{Domain} rappresenta i dominî delle
variabili, i cui possibili valori sono rappresentati da oggetti della classe
\texttt{Value} e sono contenuti in un {\tt HashSet} (per questo motivo sono
stati ridefiniti i metodi {\tt equals()} e {\tt hashcode()}). È possibile
definire una relazione tra due dominî tramite la classe {\tt Relation} e
derivate, ovvero {\tt EqConstraint}, che rappresenta un vincolo di uguaglianza,
e {\tt DiffConstraint}, che rappresenta un vincolo di disuguaglianza. La
relazione di default tra ogni coppia di dominî per la quale non ne è stata
definita una è il loro prodotto cartesiano. Le coppie dei valori vengono
memorizzate per \emph{afterset}\footnote{Data una relazione $R$, si definisce
\emph{afterset} (o \emph{successor neighborhood}) di $x$ in $R$ l'insieme $xR = \{\
y\ |\ xRy\ \}$.}: una {\tt HashMap} mappa ogni valore del dominio della
relazione al suo afterset (un {\tt HashSet}). Tramite gli afterset vengono poi
calcolati, in base al tipo di relazione, i possibili valori che soddisfano i
vincoli, se presenti, tramite il metodo  {\tt getAdjacencySet()}.

\lstinputlisting[caption=Value.java, label=lst:val]{../../../src/dsl/Value.java}
\lstinputlisting[caption=Domain.java, label=lst:set]{../../../src/dsl/Domain.java}

\bgroup
\createlinenumber{29}{\smallvdots}
\createlinenumber{30}{31}
\begin{lstlisting}[caption=Relation.java, label=lst:rel]
package dsl;

import java.util.*;

public class Relation {
    protected final HashMap<Value, HashSet<Value>> afterSets = new HashMap<>();
    protected final Domain domain;
    protected final Domain range;

    public Relation(Domain domain, Domain range) {
        this.domain = domain;
        this.range = range;
    }

    public boolean addPair(Value lValue, Value rValue) {
        return afterSets.computeIfAbsent(lValue, k -> new HashSet<>()).add(rValue);
    }

    public HashSet<Value> getAfterSet(Value value) {
        return afterSets.getOrDefault(value, new HashSet<>());
    }

    public Domain getDomain() { return domain; }
    public Domain getRange() { return range; }

    public HashSet<Value> getAdjacencySet(Value value) {
        return range.getValues();
    }
    // See ex. 5
}
\end{lstlisting}
\egroup

\bgroup
\createlinenumber{14}{\smallvdots}
\createlinenumber{15}{19}
\begin{lstlisting}[caption=EqConstraint.java, label=lst:eq]
package dsl;

import java.util.HashSet;

public class EqConstraint extends Relation {
    public EqConstraint(Domain domain, Domain range) { super(domain, range); }

    @Override
    public HashSet<Value> getAdjacencySet(Value value) {
        HashSet<Value> afterSet = getAfterSet(value);

        return afterSet.size() > 1? new HashSet<>() : afterSet;
    }
    // See ex. 5
}
\end{lstlisting}
\egroup

\bgroup
\createlinenumber{15}{\smallvdots}
\createlinenumber{16}{25}
\begin{lstlisting}[caption=DiffConstraint.java, label=lst:diff]
package dsl;

import java.util.HashSet;

public class DiffConstraint extends Relation {
    public DiffConstraint(Domain domain, Domain range) { super(domain, range); }

    @Override
    public HashSet<Value> getAdjacencySet(Value value) {
        HashSet<Value> values = range.getValues();
        values.removeAll(getAfterSet(value));

        return values;
    }
    // See ex. 5
}
\end{lstlisting}
\egroup

% \lstinputlisting[caption=Relation.java, label=lst:rel]{../../../src/dsl/Relation.java}
% \lstinputlisting[caption=EqConstraint.java, label=lst:eq]{../../../src/dsl/EqConstraint.java}
% \lstinputlisting[caption=DiffConstraint.java, label=lst:diff]{../../../src/dsl/DiffConstraint.java}
% \lstinputlisting[caption=CartesianProduct.java, label=lst:cp]{../../../src/dsl/CartesianProduct.java}
