\section{Esercizio 1}

Sono state definite le classi: %
%
\begin{enumerate*}

    \item \texttt{Value} (\autoref{lst:val}), i cui oggetti rappresentano i
    possibili valori di un dominio. Sono stati ridefiniti i metodi {\tt
    equals()} e {\tt hashcode()} per poter utilizzare correttamente oggetti {\tt
    Value} in un {\tt HashSet} o come chiave di una {\tt HashMap};

    \item \texttt{Domain} (\autoref{lst:set}), che rappresenta i dominî delle
    variabili, i cui possibili valori ({\tt Value}) sono contenuti in un {\tt
    HashSet}. È possibile definire una relazione ({\tt Relation}) con un altro
    dominio tramite il metodo {\tt addRelation()}. La relazione di default con
    ogni altro dominio è il prodotto cartesiano ({\tt CartesianProduct});

    \item La classe astratta {\tt Relation} (\autoref{lst:rel}) e le sue
    implementazioni {\tt EqConstraint}, {\tt DiffConstraint} e {\tt
    CartesianProduct} (\autoref{lst:eq}, \ref{lst:diff} e \ref{lst:cp}), che
    rappresentano le relazioni (o i vincoli) tra un dominio ed un altro. Le
    coppie dei valori vengono memorizzate per \emph{afterset}\footnote{Data una
    relazione $R$, si definisce \emph{afterset} (o \emph{successor
    neighborhood}) di $x$ in $R$ l'insieme $xR = \{\ y\ |\ xRy\ \}$.}: una {\tt
    HashMap} mappa ogni valore del dominio della relazione al suo afterset (un
    {\tt HashSet}). Tramite gli afterset vengono poi calcolati, in base
    all'implementazione della classe concreta, i possibili valori che soddisfano
    la relazione (con il metodo astratto {\tt getAdjacencySet()}).

\end{enumerate*}

\lstinputlisting[caption=Value.java, label=lst:val]{../../../src/dsl/Value.java}
\lstinputlisting[caption=Domain.java, label=lst:set]{../../../src/dsl/Domain.java}
\lstinputlisting[caption=Relation.java, label=lst:rel]{../../../src/dsl/Relation.java}
\lstinputlisting[caption=EqConstraint.java, label=lst:eq]{../../../src/dsl/EqConstraint.java}
\lstinputlisting[caption=DiffConstraint.java, label=lst:diff]{../../../src/dsl/DiffConstraint.java}
\lstinputlisting[caption=CartesianProduct.java, label=lst:cp]{../../../src/dsl/CartesianProduct.java}
