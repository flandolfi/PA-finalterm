\section{Esercizio 2}

Il package {\tt compiler} (\autoref{lst:type}, \ref{lst:scanner}, \ref{lst:parser} e \ref{lst:cex}) implementa l'analizzatore lessicale ({\tt Scanner}) e sintattico ({\tt Parser}). Sono state prese le seguenti scelte progettuali: %
%
\begin{enumerate*}

    \item i domini devono avere un nome letterale (ad es. {\tt x}, {\tt y}, ...);

    \item i valori devono cominciare col nome del dominio a cui appartengono,
    seguito da un numero (ad es. {\tt x1}, {\tt x2}, ...);

    \item le relazioni devono avere un solo dominio e un solo codominio;

    \item le relazioni su uno stesso valore sono trattate in congiunzione logica
    (ad es. {\tt !\{(x1, y1), (x1, y2)\}} $\equiv x = x1 \Rightarrow y \neq y1
    \land y \neq y2$);

    \item non si possono definire più relazioni sugli stessi due insiemi:
    essendo il vincolo di equivalenza più restrittivo di quello di
    disuguaglianza, quest'ultimo può risultare superfluo o, peggio, creare un
    vincolo insoddisfacibile. Se invece si definiscono due relazioni dello
    stesso tipo, queste possono essere scritte come un'unica relazione.

\end{enumerate*}

\lstinputlisting[caption=Type.java, label=lst:type]{../../../src/compiler/Type.java}
\lstinputlisting[caption=Scanner.java, label=lst:scanner]{../../../src/compiler/Scanner.java}
\lstinputlisting[caption=Parser.java, label=lst:parser]{../../../src/compiler/Parser.java}
\lstinputlisting[caption=CompilerException.java, label=lst:cex]{../../../src/compiler/CompilerException.java}
