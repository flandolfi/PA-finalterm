\section*{Esercizio 2}

Il package {\tt compiler} (\autoref{lst:type}, \ref{lst:scanner},
\ref{lst:parser} e \ref{lst:cex}) implementa l'analizzatore lessicale (classe
{\tt Scanner}) e sintattico (classe {\tt Parser}). Sono state prese le seguenti
scelte progettuali: %
%
\begin{enumerate*}

    \item i dominî devono avere un nome letterale (ad es. {\tt x}, {\tt y}, ...);

    \item i valori devono cominciare col nome del dominio a cui appartengono,
    seguito da un numero (ad es. {\tt x1}, {\tt x2}, ...);

    \item le relazioni devono avere un solo dominio e un solo codominio;

    \item i vincoli su uno stesso valore sono trattati in congiunzione logica
    (ad es. {\tt !\{(x1, y1), (x1, y2)\}} $\equiv x = x1 \Rightarrow y \neq y1
    \land y \neq y2$);

    \item non si possono definire più vincoli sugli stessi due insiemi: essendo
    quello di equivalenza più restrittivo di quello di disuguaglianza,
    quest'ultimo può risultare superfluo o, peggio, creare un vincolo
    insoddisfacibile. Se invece si definiscono due relazioni dello stesso tipo,
    queste possono essere scritte come un'unica relazione.

\end{enumerate*}
%
% I vari metodi della classe {\tt Parser} implementano le produzioni della
% seguente grammatica EBNF, dove $String$ e $Number$ rappresentano le produzioni
% di termini alfabetici e numerali:
%
% \begin{minipage}{0.5\textwidth}
%     \begin{align*}
%         CSP &::= Domain^+\ Relation\text{*} \\
%         Domain &::= String\ \text{\tt =}\ \text{\tt \{}\ Value^+\ \text{\tt \}} \\
%         Value &::= String\ Number \\
%     \end{align*}
% \end{minipage}
% \begin{minipage}{0.5\textwidth}
%     \begin{align*}
%         Relation &::= \text{\tt \{}\ Pair^+\ \text{\tt \}}\ \ |\ \ \text{\tt !\{}\ Pair^+\ \text{\tt \}} \\
%         Pair &::= \text{\tt (}\ Value\ \text{\tt ,}\ Value\ \text{\tt )} \\
%         & \\
%     \end{align*}
% \end{minipage}

\lstinputlisting[caption=Type.java, label=lst:type]{../../../src/compiler/Type.java}
\lstinputlisting[caption=Scanner.java, label=lst:scanner]{../../../src/compiler/Scanner.java}
\lstinputlisting[caption=Parser.java, label=lst:parser]{../../../src/compiler/Parser.java}
\lstinputlisting[caption=CompilerException.java, label=lst:cex]{../../../src/compiler/CompilerException.java}
